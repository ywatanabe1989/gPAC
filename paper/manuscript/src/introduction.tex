%% -*- mode: latex -*-
%% Introduction for gPAC manuscript

\section{Introduction}
\label{sec:introduction}

Neural oscillations, rhythmic patterns of electrical activity in the brain, orchestrate information processing across spatial and temporal scales \citep{Buzsaki2004}. Among the various forms of neural synchronization, phase-amplitude coupling (PAC) has emerged as a fundamental mechanism linking slow and fast oscillatory dynamics \citep{Canolty2010}. PAC quantifies how the phase of low-frequency oscillations modulates the amplitude of high-frequency activity, revealing hierarchical organization in neural networks \citep{Jensen2007}. This cross-frequency coupling serves critical functions in cognition, including memory consolidation in the hippocampus \citep{Tort2008}, attention control in cortical networks \citep{Szczepanski2014}, and sensorimotor integration \citep{deHemptinne2013}.

The biological significance of PAC extends beyond basic neuroscience to clinical applications. Aberrant PAC patterns characterize numerous neurological and psychiatric disorders, including Parkinson's disease \citep{deHemptinne2015}, epilepsy \citep{Amiri2016}, and schizophrenia \citep{Kirihara2012}. These pathological signatures have motivated the development of PAC-based biomarkers for disease diagnosis and treatment monitoring. However, the computational demands of PAC analysis have limited its adoption in clinical settings where real-time processing and large-scale data analysis are essential.

Traditional PAC computation involves several computationally intensive steps: (1) bandpass filtering to isolate frequency components, (2) Hilbert transformation to extract instantaneous phase and amplitude, (3) calculation of coupling metrics such as the Modulation Index, and (4) statistical validation through surrogate data testing \citep{Tort2010}. Each step presents computational bottlenecks, particularly when analyzing high-density recordings with hundreds or thousands of channels. Current CPU-based implementations can require hours or days to process large datasets, precluding real-time applications and limiting exploratory analyses.

The emergence of Graphics Processing Units (GPUs) as general-purpose computing platforms offers a solution to these computational challenges. GPUs excel at parallel processing tasks, making them ideal for the inherently parallelizable operations in PAC analysis. Previous work has demonstrated GPU acceleration for specific neuroscience applications \citep{Pachitariu2016}, but no comprehensive GPU-accelerated PAC framework has been developed. Furthermore, the integration of PAC analysis with modern deep learning frameworks remains unexplored, despite the potential for end-to-end optimization of analysis parameters.

Here we present gPAC, a GPU-accelerated Python package that dramatically accelerates PAC computation while introducing novel capabilities for neural data analysis. Our framework leverages PyTorch's tensor operations and automatic differentiation to achieve 100-1000× speedup compared to CPU implementations. Beyond performance improvements, gPAC introduces trainable frequency filters that enable data-driven optimization of frequency bands, a critical advancement given the ongoing debate about optimal frequency ranges for PAC analysis \citep{Aru2015}.

We demonstrate gPAC's capabilities through comprehensive benchmarks on synthetic and real neural data, showing linear scaling with data size and efficient multi-GPU utilization. Our validation studies confirm numerical accuracy compared to established methods while revealing the impact of implementation choices on PAC estimates. We further showcase novel applications enabled by GPU acceleration, including real-time PAC visualization and large-scale connectivity analyses previously infeasible with CPU-based methods.

By releasing gPAC as an open-source package, we aim to democratize access to high-performance PAC analysis and accelerate discoveries in systems neuroscience. The framework's modular design and PyTorch integration facilitate custom extensions and integration with existing analysis pipelines. We envision gPAC enabling new research directions in understanding cross-frequency dynamics and translating PAC-based biomarkers to clinical practice.