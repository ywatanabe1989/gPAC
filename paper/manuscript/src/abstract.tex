%% -*- mode: latex -*-
%% Abstract for gPAC manuscript

\begin{abstract}
  \pdfbookmark[1]{Abstract}{abstract}
Phase-amplitude coupling (PAC) quantifies cross-frequency interactions in neural oscillations and serves as a biomarker for cognitive processes and neurological disorders. However, traditional PAC analysis methods are computationally intensive, limiting their application to large-scale neural datasets. Here we present gPAC, a GPU-accelerated Python package that leverages PyTorch's parallel processing capabilities to dramatically accelerate PAC computation. Through optimized algorithms including vectorized bandpass filtering, parallel Hilbert transforms, and efficient surrogate generation for statistical testing, gPAC achieves 100-1000× speedup compared to CPU-based implementations while maintaining numerical accuracy (correlation > 0.99 with established methods). The framework supports multi-GPU scaling, enabling real-time analysis of high-density recordings with thousands of channels. We demonstrate gPAC's performance on synthetic and real neural data, showing linear scaling with data size and efficient memory usage even for terabyte-scale datasets. Beyond speed improvements, gPAC introduces novel features including trainable frequency filters for deep learning integration, comprehensive statistical testing with optimized permutation methods, and seamless interoperability with existing neuroscience toolboxes. Our open-source package (available at \url{https://pypi.org/project/gpu-pac/}) democratizes large-scale PAC analysis, enabling neuroscientists to explore cross-frequency dynamics in unprecedented detail and potentially accelerating discoveries in systems neuroscience and clinical applications.

\end{abstract}
